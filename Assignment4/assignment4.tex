%\documentclass{article}
\documentclass[a4paper,12pt]{article}
% Seitenränder in schön für Steven
\usepackage[paper=a4paper,left=25mm,right=25mm,top=25mm,bottom=25mm]{geometry}
\usepackage{enumitem}
\usepackage{amsmath}
\usepackage{float}
\usepackage{graphicx}
\usepackage{tikz}
\usepackage{titling}

\floatstyle{boxed}
\restylefloat{figure}

% Schusterjungen und Hurenkinder bestrafen
\clubpenalty50000
\widowpenalty50000
\displaywidowpenalty=50000

% Buchstaben mit kringel drum: %
\newcommand*\mycirc[1]{%
	\begin{tikzpicture}[baseline=(C.base)]
	\node[draw,circle,inner sep=1pt](C) {#1};
	\end{tikzpicture}}

\author{Benedikt Hans, Christoph Dollase, Steven Te\ss endorf}
\setlength{\droptitle}{-5em} % set the title to the top of the page

% ==========================
% ===== START HERE!! =======
% ==========================
\title{ \textbf{Problem Sheet 4}} % Nummer des Aufgabenblattes
\setcounter{section}{4} % Nummer des Aufgabenblattes

\begin{document}	 
	\maketitle	 %Some Vodoo-magic
	
\subsection{Baud}
\textbf{What is the difference between Baud und Bit/s?}\\
\\
\textit{Baud} is a measure comming from telecomunications and is older then the digital measure of Bit/s.
One Baud means that there has been sent one symbol per second ($1Bd = 1\frac{symbols}{second}$). The data rate depends of the encoding of the signal. If the signal is binary baud is equal to bit/s. But with multilevel signals the conversion differs.\\
\\
\textbf{A quaternary digital signal has a symbol rate of v = 106 Baud. Is the data rate equal,	smaller or greater than the Symbol rate? Is this allways the case?}\\
\\
Like discribed above the data rate depends on the signal. With a quaternary signal (four possible values) the data rate is double as high as the baud measure. \\
with $n$ being the number of possible values of the signal, the data rate is calculated as follows: \\
data rate in bps = $v \cdot log_{2}(n)$ \\
\\ 
\textbf{For the given signal, what is the data rate?}\\
\\
For $v = 106 Bd$ and a signal with $n=4$ possible values (quaternary):\\
data rate = $106 \cdot log_{2}(4)$ = $106 \cdot 2$ \\
data rate = 212 bps


\subsection{Multilevel Signals}
\textbf{Represent the following sequence of bits as a quaternary signal with a baud rate of 5/s in a time-voltage-diagram: 00011011001110011010. Determine the bitrate.} See figure \ref{fig:timevoltdia}.

\begin{figure}[H]
	\centering
	\begin{tikzpicture}
		%Basis: grid, Achsen, Beschriftung
		\draw[lightgray] (0,-2.5) grid [xstep=1cm,ystep=1cm] (11,2.5); %some help grid lines for orientation 
		\draw[thick, ->] (0,0) -- (11,0);
		\draw[thick, <->] (0,-2.5) -- (0,2.5);
		\node [right] at (11,0) {$t$};
		\node [align=center,above] at (0,2.5) {signal amplitude \\ (in Volt)};
		\node [left] at (0,2) {\textcolor{blue}{$[11]$}~~ 2};
		\node [left] at (0,1) {\textcolor{blue}{$[10]$}~~ 1};
		\node [left] at (0,-1) {\textcolor{blue}{$[01]$}~ -1};
		\node [left] at (0,-2) {\textcolor{blue}{$[00]$}~ -2};
		% egentliche Pfad für den Code:
		% 00 01 10 11 00 11 10 01 10 10
		% Codierung in Volt:
		% -2 -1  1  2 -2  2  1 -1  1  1
		\draw[line width=2pt, red] (0,-2) -- (1,-2) -- (1,-1) -- (2,-1) -- (2,1) -- (3,1) -- (3,2) -- (4,2) -- (4,-2) -- (5,-2) --  (5,2) -- (6,2) -- (6,1) -- (7,1) -- (7,-1) -- (8,-1) -- (8,1) -- (9,1)-- (10,1);
		% Legende aufzählung der bits:
		\node [align=center, below left] at (0,-2.5) {code\\step\\time};
		\node [align=center,below] at (0.5,-2.5)
			{\textcolor{blue}{$[00]$} \\ 1};
		\node [align=center,below] at (1.5,-2.5)
			{\textcolor{blue}{$[01]$} \\ 2};
		\node [align=center,below] at (2.5,-2.5)
			{\textcolor{blue}{$[10]$} \\ 3};
		\node [align=center,below] at (3.5,-2.5)
			{\textcolor{blue}{$[11]$} \\ 4};
		\node [align=center,below] at (4.5,-2.5)
			{\textcolor{blue}{$[00]$} \\ 5 \\1s};
		\node [align=center,below] at (5.5,-2.5)
			{\textcolor{blue}{$[11]$} \\ 6};
		\node [align=center,below] at (6.5,-2.5)
			{\textcolor{blue}{$[10]$} \\ 7};
		\node [align=center,below] at (7.5,-2.5)
			{\textcolor{blue}{$[01]$} \\ 8};
		\node [align=center,below] at (8.5,-2.5)
			{\textcolor{blue}{$[10]$} \\ 9};
		\node [align=center,below] at (9.5,-2.5)
			{\textcolor{blue}{$[10]$} \\ 10 \\2s};
		\node [align=center, below right] at (0,-4.5)
			{data rate = $5 \cdot log_{2}(4)$ = $10 bps$ (with baud rate of $5/s$)};
	\end{tikzpicture}
	\caption{time-voltage-diagram for bit sequence \textcolor{blue}{$[00011011001110011010]$}.}
	\label{fig:timevoltdia}
\end{figure}

	
\subsection{Channel Bandwidth}
\textbf{Consider a channel to have a bandwidth of 4 MHz. How many bit per second can be sent if digital signals with 8 levels are used? The channel shall be noiseles}

%SOLUTION HERE!!!	
	
\subsection{Signal to Noise Ratio}
\textbf{A binary signal is sent via a 3 kHz wide channel with a signal to noise ratio of 20 dB. Calculate the maximum data rate.}

%SOLUTION HERE!!!	

\subsection{Noiseless channel}
\textbf{Specify the maximum data rate that can be achieved over a noiseless 4 kHz wide channel.}

%SOLUTION HERE!!!	

\subsection{Base- and Broadband}
\textbf{Explain the term baseband and broadband. Why do we need broadband communication? Explain how broadband communication of baseband signals is achieved. Give example application scenarios.}

%SOLUTION HERE!!!	


\end{document}

% Hier nach passiert nichts mehr, daher nutzen wir das als kleines Cheat-Sheet ;)
% ===============================================================================

% Aufzählungen (auch merhstufig):
\begin{itemize}[itemsep=0pt]
	\item 
\end{itemize}

%Bilder eifnügen:
\begin{figure}[h!] %h! sorgt dafür dass das Bild möglichst nicht woanders hingeschoben wird
	%Erklärung: [width=0.5\linewidth] -> Bild ist maximal so breit wie die Hälfte des Schriftbildes
	\includegraphics[width=0.5\linewidth]{Bildname.jpg} 
	\caption{Bildunterschrift}
\end{figure}

%Tabelle einfügen:
\begin{table}[h!] %h! sorgt dafür dass die Tabelle möglichst nicht woanders hingeschoben wird
	\caption{Tabellenüberschrift}
	%hinter {tabular}: Anzahl Spalten (c=center, l=linksbündig, r=rechtsbündig, | Spaltenstriche)
	\begin{tabular}{|c|c|c} 
		A & B & C  \\ % \\ = return (neue zeile)
		\hline % horinzontale Linie
		0 & 1 & 2
	\end{tabular}
\end{table}
