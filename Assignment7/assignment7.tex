%\documentclass{article}
\documentclass[a4paper,12pt]{article}
% Seitenränder in schön für Steven
\usepackage[paper=a4paper,left=25mm,right=25mm,top=25mm,bottom=25mm]{geometry}
\usepackage{enumitem}
\usepackage{amsmath}
\usepackage{float}
\usepackage{graphicx}
\usepackage{tikz}
\usepackage{titling}
\usepackage{ amssymb }

% Schusterjungen und Hurenkinder bestrafen
\clubpenalty50000
\widowpenalty50000
\displaywidowpenalty=50000

% Buchstaben mit kringel drum: %
\newcommand*\mycirc[1]{%
	\begin{tikzpicture}[baseline=(C.base)]
	\node[draw,circle,inner sep=1pt](C) {#1};
	\end{tikzpicture}}


\author{Benedict Hans, Christoph Dollase, Steven Te\ss endorf}
\setlength{\droptitle}{-5em} % set the title to the top of the page

% ==========================
% ===== START HERE!! =======
% ==========================
\title{ \textbf{Problem Sheet 7}}
\setcounter{section}{7} % Nummer des Aufgabenblattes

\begin{document}	 
	\maketitle	 %Some Vodoo-magic
	
	\subsection{Flow Control}
	\textbf{Repeat and discuss the task of flow control}
	
	\begin{itemize}
		\item How to prevent a slow receiver swamped?
		\item Feedback-based flow control
		\begin{itemize}
			\item Receiver sends back information to the sender giving permission to send more data
		\end{itemize}
		\item Rate-based flow control
		\begin{itemize}
			\item Protocol limits the rate of data a sender may transmit without feedback from the 
			receiver
		\end{itemize}
	\end{itemize}
	
	\subsection{Transmission Errors}
	\textbf{In some layer 2 protocols, the Data Link Layer requests damaged frames to be send again by the sender. Assume that the probability of a frame to be damaged is $p$, and that acknowledgements are never lost.  How often has a frame to be transmitted on average?}
	
	
	% Solution
	
	\subsection{Error correction by parity}
	\textbf{One simple mechanism for correcting errors is to send data blockwise, with	$n$ rows and $k$ bit per row, adding one parity bit to each column and one each row (double parity).  Can this code correct every possible single-bit error? How about two-bit burst errors?}
	
	\begin{figure}[h!]
		\begin{minipage}{0.3\linewidth}
			\textbf{transmitter}\\
			\begin{tabular}{c|c}
				 bits & parity bit \\ \hline
				 1011 & 1 \\
				 0100 & 1 \\
				 1001 & 0 \\
				 0101 & 0 \\ \hline
				 \textcolor{blue}{0011} & \\ 
			\end{tabular}
		\end{minipage}
	\hfill
	\begin{minipage}{0.3\linewidth}
		\textbf{receiver}\\
		\begin{tabular}{c|c}
			bits & parity bit \\ \hline
			\textcolor{pink}{10\textcolor{red}{0}1} & \textcolor{pink}{1} \textcolor{green}{\checkmark} \\
			\textcolor{pink}{01\textcolor{red}{1}0} & \textcolor{pink}{1} \textcolor{green}{\checkmark} \\
			00\textcolor{orange}{1}1 & 0 \textcolor{red}{$\times$} \\
			01\textcolor{orange}{0}1 & 0 ~~ \\ \hline
			\textcolor{blue}{00\textcolor{orange}{1}1} & \\ 
		\end{tabular}
	\end{minipage}
	\hfill
	\begin{minipage}{0.36\linewidth}
		the double-parity helps only if there is one error. if there are errors that interfere each other such as they make the parity bit correct, it's imposssible to detect the error
	\end{minipage}

	\end{figure}
	
	\subsection{Cyclic Redundancy Checksum}
	\textbf{Determine if the following bit sequences have been transmitted error-free.  Use the generator polynomial G	to apply a CRC check. If you find an error in a bit sequence,  assume that the checksum bits contain the error. Recalculate the checksum\\}
		
	\begin{figure}[h!]
		\begin{minipage}{0.49\linewidth}
			\centering
			\begin{tabular}{c|l}\hline
				~ & Data + CRC \\ \hline
				1. & 0111 0110 0010 \\
				2. & 0101 0010 1001 \\
				3. & 1111 0110 1100 \\ \hline
			\end{tabular}
		\end{minipage}
		\hfill
		\begin{minipage}{0.49\linewidth}
			\begin{equation*}
			G(x) = x^4 + x + 1		
			\end{equation*}
		\end{minipage}	
	\end{figure}.


		$\rightarrow$	$G(x) = \textcolor{blue}{1}x^4 + \textcolor{blue}{0}x^3 + \textcolor{blue}{0}x^2 + \textcolor{blue}{1}x^1 + \textcolor{blue}{1}x^0$

	$\rightarrow$ 10011 
	\begin{table}[h!]
		\begin{tabular}{c|c|c|l}\hline
			~ & Data + CRC & (Data + CRC) : 10011 & result \\ \hline
			1. & 0111 0110 0010 & 000000 & no error \\
			2. & 0101 0010 1001 & 0001101 & error \\
			3. & 1111 0110 1100 & 0000000 & no error\\ \hline
			\multicolumn{3}{2}{recalculate checksum for 2.} \\ \hline
			2. & 0101 0010 0000 & 0000100 & error \\
		\end{tabular}
	\end{table}
	
	
	\subsection{Bit Stuffing}
	\begin{enumerate}[label=(\roman*),itemsep=0pt]
		\item \textbf{Frame the following bit sequence with bit stuffing. Use the framing pattern 011110 on $0011~0111~0111~1011~1111$}
		\begin{itemize}
			\item % SOLUTION
		\end{itemize}
		\item \textbf{Undo the framing in the following bit sequence using the same framing pattern. $0111~1000~ 1100~ 0111~ 1011~ 1010~ 1111~ 0100~ 0001~ 1001~ 1110$}
		\begin{itemize}
			\item % SOLUTION
		\end{itemize}
	\end{enumerate}


\end{document}

% Hier nach passiert nichts mehr, daher nutzen wir das als kleines Cheat-Sheet ;)
% ===============================================================================

% Aufzählungen (auch merhstufig):
\begin{itemize}[itemsep=0pt]
	\item 
\end{itemize}

%Bilder eifnügen:
\begin{figure}[h!] %h! sorgt dafür dass das Bild möglichst nicht woanders hingeschoben wird
	%Erklärung: [width=0.5\linewidth] -> Bild ist maximal so breit wie die Hälfte des Schriftbildes
	\includegraphics[width=0.5\linewidth]{Bildname.jpg} 
	\caption{Bildunterschrift}
\end{figure}

%Tabelle einfügen:
\begin{table}[h!] %h! sorgt dafür dass die Tabelle möglichst nicht woanders hingeschoben wird
	\caption{Tabellenüberschrift}
	%hinter {tabular}: Anzahl Spalten (c=center, l=linksbündig, r=rechtsbündig, | Spaltenstriche)
	\begin{tabular}{|c|c|c} 
		A & B & C  \\ % \\ = return (neue zeile)
		\hline % horinzontale Linie
		0 & 1 & 2
	\end{tabular}
\end{table}
