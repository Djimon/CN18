%\documentclass{article}
\documentclass[a4paper,12pt]{article}
% Seitenränder in schön für Steven
\usepackage[paper=a4paper,left=25mm,right=25mm,top=25mm,bottom=25mm]{geometry}
\usepackage{enumitem}
\usepackage{amsmath}
\usepackage{float}
\usepackage{graphicx}
\usepackage{tikz}
\usepackage{titling}
\usepackage{setspace}


% Schusterjungen und Hurenkinder bestrafen
\clubpenalty50000
\widowpenalty50000
\displaywidowpenalty=50000

% Buchstaben mit kringel drum: %
\newcommand*\mycirc[1]{%
	\begin{tikzpicture}[baseline=(C.base)]
	\node[draw,circle,inner sep=1pt](C) {#1};
	\end{tikzpicture}}

\newcommand*\red[1]{\textcolor{red}{#1}}

\author{Benedict Hans, Christoph Dollase, Steven Te\ss endorf}
\setlength{\droptitle}{-5em} % set the title to the top of the page

% ==========================
% ===== START HERE!! =======
% ==========================
\title{ \textbf{Problem Sheet 9}}
\setcounter{section}{9} % Nummer des Aufgabenblattes

\begin{document}	 
	\maketitle	 %Some Vodoo-magic
	
	\subsection{?}
	
	\subsection{?}
	
	\subsection{Sliding Window - Tetransmission Strategies}
	\textbf{List and discuss the advantages and disadvantages of the three retransmission strategies introduced in the lecture.}
	
	\begin{table}[h!]
		\caption{Pipelining and Go-back-N}
		\begin{tabular}{p{0.45\linewidth}|p{0.45\linewidth}}
			advantages & disadvantages \\ \hline
			$\bullet$ sender can send many frames at a time & $\bullet$ requires buffer \\ 
			$\bullet$ sender can be set for a groub of frames & $\bullet$ transmitter needs to store the last N packets\\
			$\bullet$ one ACK can acknowledge mire than one frame & $\bullet$ scheme is inefficient when there is a large delay/ high transmission rate\\
			$\bullet$ higher frequency & $\bullet$ unnecessary retransmission on many error-free packets \\			
		\end{tabular}
	\end{table}

	\begin{table}[h!]
		\caption{Pipelining and Selectvie Repeat}
		\begin{tabular}{p{0.45\linewidth}|p{0.45\linewidth}}
			advantages & disadvantages \\ \hline
			$\bullet$ fewer retransmisions &  $\bullet$ more complex\\
			$\bullet$ better performance & $\bullet$ each frame must be acknowledged individually\\
			$\bullet$ receiver re-achnowledges already received packets with certain sequence numbers below the current window case, if the receiver does not do so, then the sender's window would not move forward & $\bullet$ receiver may receive frames out of seuqence  \red{why?} \\
		\end{tabular}
	\end{table}

	\begin{table}[h!]
		\caption{Pipelining and Selective Reject (''Selective Retransmission"}
		\begin{tabular}{p{0.45\linewidth}|p{0.45\linewidth}}
			advantages & disadvantages \\ \hline
			$\bullet$ minimized retransmissions &  $\bullet$ needs extra logic for reinserting retransmitted frame in proper sequence\\
		\end{tabular}
	\end{table}


	
\end{document}

% Hier nach passiert nichts mehr, daher nutzen wir das als kleines Cheat-Sheet ;)
% ===============================================================================

% Aufzählungen (auch merhstufig):
\begin{itemize}[itemsep=0pt]
	\item 
\end{itemize}

%Bilder eifnügen:
\begin{figure}[h!] %h! sorgt dafür dass das Bild möglichst nicht woanders hingeschoben wird
	%Erklärung: [width=0.5\linewidth] -> Bild ist maximal so breit wie die Hälfte des Schriftbildes
	\includegraphics[width=0.5\linewidth]{Bildname.jpg} 
	\caption{Bildunterschrift}
\end{figure}

%Tabelle einfügen:
\begin{table}[h!] %h! sorgt dafür dass die Tabelle möglichst nicht woanders hingeschoben wird
	\caption{Tabellenüberschrift}
	%hinter {tabular}: Anzahl Spalten (c=center, l=linksbündig, r=rechtsbündig, | Spaltenstriche)
	\begin{tabular}{|c|c|c} 
		A & B & C  \\ % \\ = return (neue zeile)
		\hline % horinzontale Linie
		0 & 1 & 2
	\end{tabular}
\end{table}
