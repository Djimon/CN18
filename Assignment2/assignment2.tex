%\documentclass{article}
\documentclass[a4paper,12pt]{article}
\usepackage{enumitem}
\usepackage{amsmath}
\usepackage{graphicx}
\usepackage{tikz}
\usepackage{titling}
\usepackage{geometry}
%Seitenränder in schön für Steven
\geometry{a4paper,left=20mm,right=20mm, top=20mm, bottom=25mm} 

% Schusterjungen und Hurenkinder bestrafen
\clubpenalty50000
\widowpenalty50000
\displaywidowpenalty=50000

% Buchstaben mit kringel drum: %
\newcommand*\mycirc[1]{%
	\begin{tikzpicture}[baseline=(C.base)]
	\node[draw,circle,inner sep=1pt](C) {#1};
	\end{tikzpicture}}

\author{Benedikt Hans, Christoph Dollase, Steven Te\ss endorf}
\setlength{\droptitle}{-5em} % set the title to the top of the page

\title{ \textbf{Problem Sheet 2}}
\setcounter{section}{2} % set assignment number

\begin{document}	 
 \maketitle	 
 
 \subsection{OSI model}
 \paragraph{Name the layers of the OSI model starting from the top and their general (most important) functions.}
 \paragraph{answer:}
\begin{itemize}
	\begin{minipage}[t]{0.48\linewidth}
		\item Application layer
		\begin{itemize}
			\item common services for the end user
			\item simple mail transfer protocol
			\item file transfer
			\item web surfing
			\item web chat
			\item email clients
			\item network data sharing
			\item virtual terminals
			\item various file and data operations 
		\end{itemize}
	\end{minipage}
	\hfill
	\begin{minipage}[t]{0.48\linewidth}
			\item Presentation layer
		\begin{itemize}
			\item  rarely implemented
			\item  data encryption / decryption
			\item  character / string conversion
			\item  data complression
			\item  graphic handling
			\item  also called syntax layer in some cases
		\end{itemize}
	\end{minipage}\\

	\item Session layer
	\begin{itemize}
		\item  rarely implemented
		\item  manages session by initiating opening \& closing of sessions between end-user application processes
		\item  controls single or multiple connections for each end-user application
		\item  directly communicates with presantation and transport layers
	\end{itemize}

	\item Transport layer
	\begin{itemize}
		\item  connection-oriented communication
		\begin{itemize}
			\item ensures that a robust connection is established before data is exchanged
		\end{itemize}
		\item  same order delivery
		\begin{itemize}
			\item  ensures that packages are delivered in a strict sequence
		\end{itemize}
		\item  data integrity
		\item  flow control
		\begin{itemize}
			\item  buffering if devices at each end of a network have different network capabilities in terms of data throughput 
		\end{itemize}	
		\item  traffic control
		\begin{itemize}
			\item  speed restrictions \& stuff
		\end{itemize}	
		\item  multiplexing
		\item  byte orientation
		\begin{itemize}
			\item if applications prefer to receive byte streams instead of packets 
		\end{itemize}  
	\end{itemize}

	\item Network layer
	\begin{itemize}
		\item  addressing and routing of packets
		\item  selects and manages the best logical path for data transfer between nodes
	\end{itemize}

	\item Data Link layer
	\begin{itemize}
		\item  transfer data frames between nodes over a network
		\item  data frame includes: source / destination addresses, data length, start signal or indicator and other related ethernet information to enhance communication		 
	\end{itemize}

	\item Physical layer
	\begin{itemize}
		\item  modulates process of converting a signal from one form to another to ensure the physical transmission
		\item  bit-by-bit delivery
		\item  bit synchronization
		\item  start-stop signaling and flow control in asynchronous serial communication
		\item  bit interleaving to improve error correction
		\item  transmission mode control
	\end{itemize}

\end{itemize}
 
 \subsection{Path selection}
 \paragraph{Which layer has the task to select the path/route in a network?}
 \paragraph{answer:} 
 $\rightarrow$ Network layer \\
 Its main responsibility is the data transmission over large distances and between heterogeneous sub-networks. This includes the task of routing, where the layer has to select a path through the network. Routers manage the tables with routing information and choose the best part th the receiver with the help of the addresses. 
 
 \subsection{Overhead}
 \paragraph{Explain which service type has more overhead: connection-oriented or connection-less-communication?} 
 \paragraph{answer:} 
 $\rightarrow$ Connection-oriented\\
 Communication has more overhead, because sending packets requires more parameters in the header of the packet to ensure the reliable transmission. Connection-less communication has a smaller packet header size, therefore it has less overhead. 
 
 
 \subsection{Connection Properties}
 \paragraph{Explain the terms simplex, duplex and half-duplex. Name an example medium of each type}
 \paragraph{answer:}
 \begin{itemize}
 	\item Simplex
 	 \begin{itemize}
 		\item  sends information in one direction only
 		\item  ex: pager, sensors, fire detector
 	\end{itemize}
 
 	\item Duplex
 	\begin{itemize}
 		\item  bidirectional communication system
 		\item  allows both end nodes to send and receive communication data or signals, simultaneously and one at a time
 		\item  both nodes are able to operate as sender and receiver at the same time
 		\item  in this case: sends and receives simulataneously, called 'full duplex'
 		\item  ex: Telephone
 	\end{itemize}
 
 	\item Half-Duplex
 	\begin{itemize}
 		\item  can send or receive, one path at a time
 		\item  ex: Walkie-Talkie, partly GSM-Voice connections
 	\end{itemize}
 \end{itemize}
 
  \subsection{Terminology}
 \paragraph{Explain the terms signal, data and information in the context of computer networks}
 \paragraph{answer:}
 \begin{itemize}
 	\item Signal
 	\begin{itemize}
 		\item  analog or digital
 		\item  physical representation of data by spatial or timely variation of physical caracteristics
 		\item  real physical representation of an abstract representation
 	\end{itemize}
 
 	\item Data
 	\begin{itemize}
 		\item  information, that has been translated into a form that is efficient for movement or processing
 		$\rightarrow$ img slide 8 
 	\end{itemize}
 
 	\item Information
 	\begin{itemize}
 		\item  information is whatever contributes to a reduction in the uncertainty of the state of a system
 		\item  in this case, uncertainty is expressed in an objectively measurable form
 		\item  whatever is capable of causing a human mind to change its opinion about the current state of the real world
 		\item  only humans can handle information, machines can't
 	\end{itemize}
 \end{itemize}
 
  \subsection{Protocol Overhead}
 \paragraph{Consider a layered network architecture with n layers. Each protocol on a layer adds a header of h bytes. What fraction of the network bandwidth is lost due to the overhead when m bytes are generated by an application on the top-most layer}
 \paragraph{answer:}

  \textbf{In PROGRESS....} \\
  \textit{n layers \\
 every layer adds an overhead of h bytes $\rightarrow$ ???}

 
\end{document}
