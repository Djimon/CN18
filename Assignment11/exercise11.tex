%\documentclass{article}
\documentclass[a4paper,12pt]{article}
% Seitenränder in schön für Steven
\usepackage[paper=a4paper,left=25mm,right=25mm,top=25mm,bottom=25mm]{geometry}
\usepackage{enumitem}
\usepackage{amsmath}
\usepackage{float}
\usepackage{graphicx}
\usepackage{tikz}
\usepackage{titling}


% Schusterjungen und Hurenkinder bestrafen
\clubpenalty50000
\widowpenalty50000
\displaywidowpenalty=50000

% Buchstaben mit kringel drum: %
\newcommand*\mycirc[1]{%
	\begin{tikzpicture}[baseline=(C.base)]
	\node[draw,circle,inner sep=1pt](C) {#1};
	\end{tikzpicture}}

\newcommand*\red[1]{\textcolor{red}{#1}}

\author{Benedict Hans, Christoph Dollase, Steven Te\ss endorf}
\setlength{\droptitle}{-5em} % set the title to the top of the page

% ==========================
% ===== START HERE!! =======
% ==========================
\title{ \textbf{Problem Sheet 11}}
\setcounter{section}{11} % Nummer des Aufgabenblattes

\begin{document}	 
	\maketitle	 %Some Vodoo-magic
	
	\subsection{Network Components}
    \textbf{Discuss the function(-s) of the following network components: Repeater, hub, switch, bridge, router, 
    and gateway. Which "data" do they handle and on which layer of the ISO/OSI reference model do they operate?}
    \begin{itemize}[itemsep=0pt]
        \item \textbf{Repeater:} Receives signals and retransmits them. It operates on the physical layer.
        \item \textbf{Hub:} Receives a signal and splits it into multiple signals. After splitting it, it sends it
            to connected devices. It operates on the physical layer.
        \item \textbf{Switch:} Sends data to a target device after receiving them. Opens packets and searches for
            the target device. Operates on data link layer.
        \item \textbf{Bridge:} Transfers data packets between networks. A transparent bridge "knows" in which
            network devices are located. Operates on data link layer.
        \item \textbf{Router:} Opens packets and searches for the target network/device. Often used as gateway
            to different networks. Operates on Network Layer.
        \item \textbf{Gateway:} Connects to systems. Data can be edited. A Gateway can operate on all layers.
    \end{itemize}
    
    \red{Gateway erfüllt einerseits die gleichen Aufgaben wie ein Router.
    Zusätzlich muss ein gateway allerdings noch mehrere Dienste bereithalten
    die auf der Applikationsebene arbeiten müssen (CUPS, DNS, DHCP, NAT) um
    eine funktionierende Kommunikation zu ermöglichen.
    Alle Geräte die man zu Hause an einem Gateway hat haben im INternet die 
    gleiche Adresse und müssen vom Gateway ankommenden Paketen zugeordnet
    werden.}
	
	% Solution
	
	\subsection{Spanning Tree}
	\textbf{The spanning tree is an important connected graph of a computer network. On which layer of the ISO/OSI
    reference model is it created?} \newline
    The spanning tree is creates on basis of the Data Link Layer. \newline
    \newline

    \textbf{Why do we need to create a spanning tree? What is the important property?} \newline
    Two bridges connecting two LANs can get caught in a loop (due to a ring like structure) and continiously send
    data between each other without any party requiring it. A spanning tree can solve this problem. The tree 
    represents a clear hierarchy between the bridges and LANs with one root bridge. \newline
    \newpage

    \textbf{Create the spanning tree for the given network. The numbers at the edges specify the cost of each path
    ; the number in the vertices specify the switch ID.} \newline

    \begin{center}%h! sorgt dafür dass das Bild möglichst nicht woanders hingeschoben wird
	    %Erklärung: [width=0.5\linewidth] -> Bild ist maximal so breit wie die Hälfte des Schriftbildes
	    \includegraphics[width=0.5\linewidth]{tree.png} 
	    \caption{Spanning Tree}
    \end{center}
    	
\end{document}

% Hier nach passiert nichts mehr, daher nutzen wir das als kleines Cheat-Sheet ;)
% ===============================================================================

% Aufzählungen (auch merhstufig):
\begin{itemize}[itemsep=0pt]
	\item 
\end{itemize}

%Bilder eifnügen:
\begin{figure}[h!] %h! sorgt dafür dass das Bild möglichst nicht woanders hingeschoben wird
	%Erklärung: [width=0.5\linewidth] -> Bild ist maximal so breit wie die Hälfte des Schriftbildes
	\includegraphics[width=0.5\linewidth]{Bildname.jpg} 
	\caption{Bildunterschrift}
\end{figure}

%Tabelle einfügen:
\begin{table}[h!] %h! sorgt dafür dass die Tabelle möglichst nicht woanders hingeschoben wird
	\caption{Tabellenüberschrift}
	%hinter {tabular}: Anzahl Spalten (c=center, l=linksbündig, r=rechtsbündig, | Spaltenstriche)
	\begin{tabular}{|c|c|c} 
		A & B & C  \\ % \\ = return (neue zeile)
		\hline % horinzontale Linie
		0 & 1 & 2
	\end{tabular}
\end{table}
