%\documentclass{article}
\documentclass[a4paper,12pt]{article}
% Seitenränder in schön für Steven
\usepackage[paper=a4paper,left=25mm,right=25mm,top=25mm,bottom=25mm]{geometry}
\usepackage{enumitem}
\usepackage{amsmath}
\usepackage{float}
\usepackage{graphicx}
\usepackage{tikz}
\usepackage{titling}


% Schusterjungen und Hurenkinder bestrafen
\clubpenalty50000
\widowpenalty50000
\displaywidowpenalty=50000

% Buchstaben mit kringel drum: %
\newcommand*\mycirc[1]{%
	\begin{tikzpicture}[baseline=(C.base)]
	\node[draw,circle,inner sep=1pt](C) {#1};
	\end{tikzpicture}}

\newcommand*\red[1]{\textcolor{red}{#1}}

\author{Benedict Hans, Christoph Dollase, Steven Te\ss endorf}
\setlength{\droptitle}{-5em} % set the title to the top of the page

% ==========================
% ===== START HERE!! =======
% ==========================
\title{ \textbf{Problem Sheet 10}}
\setcounter{section}{10} % Nummer des Aufgabenblattes

\begin{document}	 
	\maketitle	 %Some Vodoo-magic
	
	\subsection{CSMA/CD}
	\begin{enumerate}[label=(\roman*),itemsep=0pt]
		\item \textbf{What is a collision?}
		\begin{itemize}[itemsep=0pt]
			\item 
		\end{itemize}
		\item \textbf{What does Carrier Sense mean?}
		\begin{itemize}[itemsep=0pt]
			\item 
		\end{itemize}
		\item \textbf{What does thee term persistence mean?}
		\begin{itemize}[itemsep=0pt]
			\item 
		\end{itemize}
		\item \textbf{Consider a channel with only a few stations that all have a low sending propability. Which type of persistence would be most effective?}
		\begin{itemize}[itemsep=0pt]
			\item 
		\end{itemize}
		\item \textbf{Consider a channel with a huge number of stations that transmit frequently. Which type of persistence would lead to better channel utilization?}
		\begin{itemize}[itemsep=0pt]
			\item 
		\end{itemize}
	\end{enumerate}
	
	\subsection{Binary Exponential Backoff}
	\textbf{In the Ethernet specification (IEEE  802.3) a collision causes the Binary Exponential Backoff algorithm to start, in order to resolve the conflict. The algorithm is specified in the specificationas follows:
	\textit{The delay is an integral multiple of the slot time. The number of slot times to delay before the n-th retransmission attempt is chosen as a uniformly distributed random integer r in the range	$0 <= r <= 2^K$  where $K = min(n, 10)$.} In an IEEE 802.3 LAN with 4 stations the following transmission attempts (slots) are given: \\
	\textcolor{blue}{Station A: 1, 3, 12 |  Station B: 1, 7, 8 | Station C: 5 | Station D: 5 }\\
	Frames are buffered till the transmission was successful. The stations are using a function for generating random numbers for the waiting time, which produces values between 0 and 16383. Modulo division is used for restricting the values to the current interval - e.g. after the second conflict it is divided modulo 4, after the third collision modulo 8, etc. The stations’ sequences of random values are given as follows (not all of them are needed here):
	\begin{itemize}[itemsep=0pt]
		\item Station A: 394, 5453, 13815, 4410, 2883, 6402
		\item Station B: 777, 2407, 9599, 3037, 5034, 99
		\item Station C: 1258, 3547, 733, 688, 9234, 2487
		\item Station D: 944, 386, 7427, 4434, 2348, 4287
	\end{itemize}
	Fill in the following table by marking the time slots with an S for successful sending attempts, with an X for a collision, with a W if a station has to wait because of its waiting time, and with a - if a station is inactive, i.e. it has nothing to send.}

	\begin{table}[h!]
		\begin{tabular}{|l|*{20}{p{0.3cm}|}} \hline
			\textbf{Slot} &   1 & 2 & 3 & 4 & 5 & 6 & 7 & 8 & 9 & 10 & 11 & 12 & 13 & 14 & 15 & 16 & 17 & 18 & 19 & 20 \\ \hline
			Station A &    &   &   &   &   &   &   &   &   &    &    &    &    &    &    &    &    &    &    &  \\ \hline
			Station B &    &   &   &   &   &   &   &   &   &    &    &    &    &    &    &    &    &    &    &  \\ \hline
			Station C &    &   &   &   &   &   &   &   &   &    &    &    &    &    &    &    &    &    &    &  \\ \hline
			Station D &    &   &   &   &   &   &   &   &   &    &    &    &    &    &    &    &    &    &    &  \\ \hline
		\end{tabular}
	\end{table}

	
	\subsection{Collision Detection and Frame Size}
	\textbf{Consider a 10 MBit/s CSMA/CD LAN with a bus topology of 50 m. The signal propagates with $2 \cdot 10^8$ m/s in the medium. Calculate the upper bound of the collision detection time- Calculate the minimum frame length that is required to detect a collision.}
	
	
	% Solution 
	
	\subsection{Multiple Access Effectiveness}
	\textbf{Consider a bus with ten stations. Construct scenarios where the following Multiple Access strategies are efficient or inefficient, respectively, in terms of medium utilization.}
		\begin{enumerate}[label=(\roman*),itemsep=0pt]
		\item \textbf{TDMA}
		\begin{itemize}[itemsep=0pt]
			\item 
		\end{itemize}
		\item \textbf{CSMA/CD}
		\begin{itemize}[itemsep=0pt]
			\item 
		\end{itemize}
		\item \textbf{Bitmap Protocol}
		\begin{itemize}[itemsep=0pt]
			\item 
		\end{itemize}
	\end{enumerate}
	
\end{document}

% Hier nach passiert nichts mehr, daher nutzen wir das als kleines Cheat-Sheet ;)
% ===============================================================================

% Aufzählungen (auch merhstufig):
\begin{itemize}[itemsep=0pt]
	\item 
\end{itemize}

%Bilder eifnügen:
\begin{figure}[h!] %h! sorgt dafür dass das Bild möglichst nicht woanders hingeschoben wird
	%Erklärung: [width=0.5\linewidth] -> Bild ist maximal so breit wie die Hälfte des Schriftbildes
	\includegraphics[width=0.5\linewidth]{Bildname.jpg} 
	\caption{Bildunterschrift}
\end{figure}

%Tabelle einfügen:
\begin{table}[h!] %h! sorgt dafür dass die Tabelle möglichst nicht woanders hingeschoben wird
	\caption{Tabellenüberschrift}
	%hinter {tabular}: Anzahl Spalten (c=center, l=linksbündig, r=rechtsbündig, | Spaltenstriche)
	\begin{tabular}{|c|c|c} 
		A & B & C  \\ % \\ = return (neue zeile)
		\hline % horinzontale Linie
		0 & 1 & 2
	\end{tabular}
\end{table}
